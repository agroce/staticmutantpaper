\section{Conclusions and Future Work}

In this paper, we showed that program mutants can be used as a proxy
for real faults, to compare (and motivate improvements to) static
analysis tools.  Mutants are attractive in that changing a mostly
correct program usually introduces a bug into it; this is the basis of
mutation testing, after all, and a large body of work supports the
claim that at least 60-70\% of mutants are fault-inducing.   This
means we can assume mutants are faulty, and escape the
ground-truth/false positive problem that makes comparing static
analysis tools so labor-intensive.  Our approach cannot identify precision
problems directly, but combined with finding counts for un-mutated
code, can compute a mutant ratio that implicates imprecise tools.  We
evaluated 9 popular static analysis tools, for Solidity smart contracts, Java,
and Python, and offer advice to users of these tools.  Our mutation
results strongly confirm the wisdom of using multiple tools; with the
exception of one Java tool, all tools we investigated uniquely
detected over 1,000 mutants, and for Java and Python there were no
mutants detected by all tools. For Solidity,
academic research evaluations of the tools generally agreed strongly
with our conclusions, but lacked the detail mutant analysis contributed.
We were also able to use our methods, plus a novel mutant prioritization scheme, to
identify and implement three useful new detectors for the open source Slither smart contract
analyzer, the best-performing of the tools.

As future work, we would like to further validate our approach and
improve our admittedly \emph{ad hoc} mutant distance metric.  Allowing
user feedback \cite{EndUserMistake,OnlyOracle}, or applying metric
learning methods \cite{kulis2012metric} (particularly unsupervised
learning \cite{scholkopf1998nonlinear,tipping1999probabilistic}) are
the most obvious and interesting possibilities for a better metric.
Finally, mutant prioritization should be applicable to improving
software testing, as well \cite{groce2018verified}.  We are also interested in combining static and traditional mutation analysis: it might be useful to consider a mutant killed if it is detected either by static or dynamic analysis, using static kills to lower the number of required costly testing/verification runs for mutants, and the combined result as a useful guide to the quality of overall \emph{defense in depth} \cite{antunes2012defending,FC20} for high-importance code.