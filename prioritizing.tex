\section{Prioritizing Mutants}
\label{sec:prioritizing}

One goal of our approach is to make it easy for tool developers to
examine cases where one tool kills a mutant and another fails to, in
order to identify patterns for new detectors.  Dedicated developers may also
simple want to scan all mutants their tool does not kill, for the same
purpose, analogous to what Groce et al. have proposed for automated
verification and testing \cite{groce2015verified,groce2018verified}.
Security analysts and other expert users who are not developers may
also wish to do this, to better understand details of tool strengths
and weaknesses.

Unfortunately the full list of unkilled mutants, or differentially
unkilled mutants, is likely to be both large and highly redundant.  In
our results below, only one of 9 tools we examined killed fewer than
1,000 mutants it alone detected.  Any cross-tool comparison is thus
likely to involve hundreds or thousands of mutants.

The problem of identifying unique ``faults'' (tool weaknesses) in this
situation is very similar to the \emph{fuzzer taming} problem in
software testing, as defined by Chen et al. \cite{PLDI13}: ``Given a
potentially large collection of test cases, each of which triggers a
bug, rank them in such a way that test cases triggering distinct bugs
are early in the list.'' \cite{PLDI13}.  Their solution was
to use Gonzalez' Furthest-Point-First \cite{Gonzalez} (FPF) algorithm
to \emph{rank} test cases so that users can examine very different
test cases as quickly as possible.  An
FPF ranking requires a distance metric $d$, and ranks items so that
dissimilar ones appear earlier.
The hypothesis of Chen et al. was that dissimilar tests, by a
well-chosen metric, will also fail due to different faults.  FPF is a
greedy algorithm that proceeds by repeatedly adding the item with the
\emph{maximum minimum distance to all previously ranked items}. Given an
initial seed item $r_0$, a set $S$ of items to rank, and a distance
metric $d$, FPF computes $r_i$ as
$s \in S: \forall s' \in S: min_{ j < i}(d(s,r_j)) \geq min_{j <
  i}(d(s',r_j))$.  The condition on $s$ is obviously true when
$s = s'$, or when $s' = r_j$ for some $j < i$; the other cases for
$s'$ force selection of \emph{some}
max-min-distance $s$.

In order to apply FPF ranking to examining mutants, we implemented a
simple, somewhat \emph{ad hoc} distance metric and FPF ranker in the
Universal Mutator \cite{universalmutator} tool.  Our metric $d$ is
the sum of a set of measurements.  First, it adds a similarity
ratio based on Levenshtein distance \cite{lev} for (1) the \emph{changes} (Levenshtein edits) from
the original source code elements to
the two mutants,  (2) the two original source code elements changed (in
general, lines), and (3) the actual output mutant code.  These are
weighted with multipliers of 5.0, 0.1, and 0.1, respectively; the type
of change (mutation operator, roughly) dominates this part of the
distance, because it best describes ``what the mutant did''; however,
because many mutants will have the same change (e.g., changing {\tt +}
to {\tt -}, the other ratios also often matter.  The Python
{\tt Levenshtein} library's similarity ratio is used, as it is based
on true minimal string edits; it reports similarity ratios between 0.0
and 1.0.

Our metric also incorporates a measure of the distance in the source
code between the locations of two mutants.  If the mutants are to
different files, this adds 0.5 to the distance; it also adds 0.25
times the number of source lines separating the two mutants if they
are in the same file, divided by 10, but caps the amount added at
0.25.  The full metric, therefore is:

$$ 5.0 \times r(\mathit{edit}_1, \mathit{edit}_2) +$$
$$0.1 \times r(\mathit{source}_1, \mathit{source}_2) +$$
$$0.1 \times r(\mathit{mutant}_1, \mathit{mutant}_2) +$$
$$0.5 \times \mathit{different\_files} +$$
$$max(0.25, \frac{\mathit{line\_distance}(\mathit{mutant}_1, \mathit{mutant}_2)}{10})$$

\noindent Where $r$ is a Levenshtein-based string similarity ratio,
and $\mathit{line\_distance}$ is the distance in a source file between
two locations, in lines (and zero if the locations are in different
source files).

We do not claim this is an optimal, or even tuned, metric; in the long run, it may be completely replaced.  It did
provide useful results for a different problem, ranking mutants in an
effort similar to that of Groce et
al. \cite{groce2015verified,groce2018verified} to examine unkilled
mutants for a smart contract library tested using the Echidna fuzzer \cite{echidna-code},
and we simply adopted it for our differential mutation analysis.
Devising a better metric is left as future work, we only wish to show
that even a hastily-devised and somewhat arbitrary metric provides
considerable advantage over wading through an un-ordered list of
mutants, and introduce the idea of using FPF for mutants, not just tests.